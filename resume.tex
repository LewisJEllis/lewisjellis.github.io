% resume.tex
% vim:set ft=tex spell:

\documentclass[10pt,letterpaper]{article}
\usepackage[letterpaper,margin=0.75in,tmargin=0.6in,bmargin=0.5in]{geometry}
\usepackage[utf8]{inputenc}
\usepackage{mdwlist}
\usepackage[T1]{fontenc}
\usepackage{textcomp}
\usepackage{tgpagella}
\usepackage{hyperref}
\hypersetup{colorlinks=true,
			urlcolor=blue}


\pagestyle{empty}
\setlength{\tabcolsep}{0em}

% indentsection style, used for sections that aren't already in lists
% that need indentation to the level of all text in the document
\newenvironment{indentsection}[1]%
{\begin{list}{}%
	{\setlength{\leftmargin}{#1}}%
	\item[]%
}
{\end{list}}

% opposite of above; bump a section back toward the left margin
\newenvironment{unindentsection}[1]%
{\begin{list}{}%
	{\setlength{\leftmargin}{-0.5#1}}%
	\item[]%
}
{\end{list}}

% format two pieces of text, one left aligned and one right aligned
\newcommand{\headerrow}[2]
{\begin{tabular*}{\linewidth}{l@{\extracolsep{\fill}}r}
	#1 &
	#2 \\
\end{tabular*}}

% make "C++" look pretty when used in text by touching up the plus signs
\newcommand{\CPP}
{C\nolinebreak[4]\hspace{-.05em}\raisebox{.22ex}{\footnotesize\bf ++}}

% and the actual content starts here
\begin{document}
\begin{center}
{\LARGE \textbf{Lewis J. Ellis}}

\vspace{0.3em}
\hspace{-1.5em}
me@LewisJEllis.com\ \ \textbullet
\ \ \href{http://lewisjellis.com}{LewisJEllis.com}
\end{center}

\hrule
\vspace{-0.7em}
\subsection*{Work Experience}

\begin{itemize}
	\parskip=0.02em

	\item
	\headerrow
		{\textbf{Shape Security}}
		{\textbf{Mountain View, CA}}
	\\
	\headerrow
		{\emph{Software Engineering Intern, KPCB Engineering Fellow}}
		{\emph{May -- August 2014}}
	\begin{itemize*}
		\item Built a robust, high-volume event collector, later used to receive and channel data for analytics.
		\item Used CasperJS to test and debug the core transformation engine \& various attack countermeasures.
		\item Built \& documented the main attack launcher component of a new R\&D verification platform, then supported other engineers and researchers using the platform to test attacks against Shape's technology.
	\end{itemize*}

	\item
	\headerrow
		{\textbf{App.net}}
		{\textbf{San Francisco, CA}}
	\\
	\headerrow
		{\emph{Software Engineering Intern}}
		{\emph{May -- August 2013}}
	\begin{itemize*}
		\item Restructured test suites to reduce build times by 15\%
		\item Migrated user search indexing from Solr to ElasticSearch, built a metric to improve place search results
		\item Designed and implemented extensive improvements to annotations on API objects, making it easier for third party apps to represent relationships between posts, users, places, files, and media.
	\end{itemize*}

	\item
	\headerrow
		{\textbf{Canada/USA Mathcamp}}
		{\textbf{Tacoma, WA}}
	\\
	\headerrow
		{\emph{Junior Counselor}}
		{\emph{June -- August 2012}}
	\begin{itemize*}
		\item Planned and excited the camp-wide hiking trip, talent show, and feedback surveys; managed camp finances and visitor accommodations; taught campers to speed cube; made lots of liquid nitrogen ice cream
	\end{itemize*}

	\item
	\headerrow
		{\textbf{University of Pennsylvania}}
		{\textbf{Philadelphia, PA}}
	\\
	\headerrow
		{\emph{Teaching Assistant \& Head Teaching Assistant}}
		{\emph{Spring 2012 -- Present}}
	\begin{itemize*}
		\item CIS 120, Programming Languages \& Techniques, with OCaml and Java, 4 semesters.
		\item CIS 121, Data Structures \& Algorithms, with proofs and Java, 1 semester.
		\item \href{http://www.seas.upenn.edu/~cis160/14fa/index.shtml}{CIS 160}, Mathematical Foundations of CS, current head TA. Lead 18 TAs, manage course website.
	\end{itemize*}

\end{itemize}

\hrule
\vspace{-0.7em}
\subsection*{Education}

\begin{itemize}
	\parskip=0.0em

	\item
	\headerrow
		{\textbf{University of Pennsylvania}}
		{\textbf{Philadelphia, PA}}
	\\
	\headerrow
		{\emph{School of Engineering and Applied Science, expected BSE May 2015. GPA: 3.6}}
		{\emph{Fall 2011 -- Present}}
	\begin{itemize*}
		\item Networked and Social Systems, a branch of CS focusing on applications of network theory.
		\item Coursework: Artificial Intelligence, Algorithms, Databases, Advanced Functional Programming, Crowdsourcing, Cloud Computing, Networked Systems, Theory of Networks, Network Security, Cryptography, Game Theory, Stochastic Systems, Optimization, Probability, Linear Algebra, Discrete Math
		\item Former developer with \href{http://pennlabs.org/}{PennLabs}, a student group dedicated to building technology for student use
		\item Organizer of \href{http://pennapps.com}{PennApps}, the premier college hackathon, hosted at Penn each semester
		\item Lead organizer of \href{http://pclassic.org}{PClassic}, a semesterly high school programming contest with 100s of participants
		\item Current \href{http://www.dolphin.upenn.edu/ultimate/void/}{Penn Ultimate} player, former \href{http://www.pennathletics.com/SportSelect.dbml?DB_OEM_ID=1700&SPID=542&SPSID=8693&DB_OEM_ID=1700}{Penn Track \& Field} long jumper
	\end{itemize*}

\end{itemize}

\hrule
\vspace{-0.7em}
\subsection*{Projects}

\begin{itemize}
	\parskip=-0.1em

	\item
	\href{https://github.com/LewisJEllis/awesome-lua}{awesome-lua} (Summer 2014). High-quality compilation of the modern Lua ecosystem with 300+ GitHub stars.
	\item
	\href{https://github.com/maxscheiber/cumulonimbus}{Cumulonimbus} (Finalist, Greylock Hackfest 2014). Seamlessly joins multiple cloud storage accounts into one.
	\item
	ABCrowd (Fall 2013). Use MTurk to enable sites/projects without the requisite traffic for effective A/B testing to get quantitative feedback. Lets users build questionnaires, then launches them and compiles the results.
	\item
	\href{https://github.com/yefim/DBIDE}{Dropbox IDE} (2nd place, HackRU Fall 2012). In-browser IDE which uses Dropbox as the user's workspace.
	\item
	PSPNet, PSPTD (2008). Text-based web browser, tower defense game, written in Lua for the PlayStation Portable. PSPTD was released to the homebrew community and downloaded over 1000 times.
\end{itemize}


\hrule
\vspace{-0.7em}
\subsection*{Skills \& Technologies}

\begin{indentsection}{\parindent}
\hyphenpenalty=1000
\begin{description*}
	\item[Languages, etc:]
	Lua, JavaScript, Python, Haskell, OCaml, Java, SQL, Redis, MongoDB, \LaTeX
	\item[Tools, platforms, etc:] Sublime, git, Node, Flask, OpenResty, AWS, MTurk, MapReduce, ElasticSearch
	\item[Familiar:] Go, C, \CPP, Ruby, Common LISP, HTML/CSS
	\item[APIs:] Experience in API design; have used App.net, Dropbox, Venmo, Twilio, Google, Facebook, Twitter APIs
	\item[Presenting:] \href{https://github.com/LewisJEllis/eslint101}{ESLint tech talk}, \href{https://github.com/LewisJEllis/nodejs-workshops}{Node.js workshop series}, 3 yrs teaching, 3-time MC of PennApps closing ceremony
\end{description*}
\end{indentsection}

\end{document}
