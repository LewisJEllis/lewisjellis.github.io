% resume.tex
% vim:set ft=tex spell:

\documentclass[10pt,letterpaper]{article}
\usepackage[letterpaper,margin=0.75in,tmargin=0.6in,bmargin=0.5in]{geometry}
\usepackage[utf8]{inputenc}
\usepackage{mdwlist}
\usepackage[T1]{fontenc}
\usepackage{textcomp}
\usepackage{tgpagella}
\usepackage{hyperref}
\hypersetup{colorlinks=true,
			urlcolor=blue}


\pagestyle{empty}
\setlength{\tabcolsep}{0em}

% indentsection style, used for sections that aren't already in lists
% that need indentation to the level of all text in the document
\newenvironment{indentsection}[1]%
{\begin{list}{}%
	{\setlength{\leftmargin}{#1}}%
	\item[]%
}
{\end{list}}

% opposite of above; bump a section back toward the left margin
\newenvironment{unindentsection}[1]%
{\begin{list}{}%
	{\setlength{\leftmargin}{-0.5#1}}%
	\item[]%
}
{\end{list}}

% format two pieces of text, one left aligned and one right aligned
\newcommand{\headerrow}[2]
{\begin{tabular*}{\linewidth}{l@{\extracolsep{\fill}}r}
	#1 &
	#2 \\
\end{tabular*}}

% make "C++" look pretty when used in text by touching up the plus signs
\newcommand{\CPP}
{C\nolinebreak[4]\hspace{-.05em}\raisebox{.22ex}{\footnotesize\bf ++}}

% and the actual content starts here
\begin{document}
\begin{center}
{\LARGE \textbf{Lewis J. Ellis}}

\vspace{0.3em}
\hspace{-1.5em}
me@LewisJEllis.com\ \ \textbullet
\ \ (208) 699-9616\ \ \textbullet
\ \ \href{http://lewisjellis.com}{LewisJEllis.com}
\end{center}

\hrule
\vspace{-0.7em}
\subsection*{Work Experience}

\begin{itemize}
	\parskip=0.02em

	\item
	\headerrow
		{\textbf{DIY Co}}
		{\textbf{San Francisco, CA}}
	\\
	\headerrow
		{\emph{Software Engineer}}
		{\emph{November 2017 - August 2019}}
	\begin{itemize*}
		\item Back-end for \href{https://DIY.org}{DIY.org} with Node.js, GraphQL, MySQL, Redis, Docker, AWS, Stripe, etc.
		\item Intro'd GraphQL, ran major data migrations, built new subscription offerings \& product feature backends.
		\item Successfully planned and executed the merge of two platforms (DIY and JAM) into one (DIY 2.0).
	\end{itemize*}


	\item
	\headerrow
		{\textbf{Sentry.io}}
		{\textbf{San Francisco, CA}}
	\\
	\headerrow
		{\emph{Software Engineer}}
		{\emph{August 2016 - June 2017}}
	\begin{itemize*}
		\item Contributed primarily to \href{https://github.com/getsentry/raven-node}{raven-node}, Sentry's error reporting SDK for Node.js, and also to \href{https://github.com/getsentry/raven-js}{raven-js}.
		\item Released \href{https://blog.sentry.io/2017/01/19/node-breadcrumbs}{1.0} \& \href{https://github.com/getsentry/raven-node/releases/tag/v2.0.0}{2.0} milestones, bringing advanced capabilities by deeply understanding the Node.js error mechanisms and the \href{https://github.com/LewisJEllis/async-context-examples}{asynchronous context problem}. Saw download numbers quadruple in six months.
	\end{itemize*}

	\item
	\headerrow
		{\textbf{Shape Security}}
		{\textbf{Mountain View, CA}}
	\\
	\headerrow
		{\emph{Software Engineer, SWE Intern (KPCB Engineering Fellow)}}
		{\emph{May -- August 2014, August 2015 - August 2016}}
	\begin{itemize*}
		\item Designed, spec'd, and implemented \href{https://github.com/shapesecurity/superpack-spec}{SuperPack} serialization format to reduce typical payloads from 20kB of JSON to 6kB of base64, using a 3kB browser JavaScript encoder; saved 20-30TB of bandwidth per month.
		\item Researched \& implemented detection and obfuscation techniques to increase detection efficacy against advanced adversary tooling like Selenium and FraudFox.
	\end{itemize*}

	\item
	\headerrow
		{\textbf{App.net}}
		{\textbf{San Francisco, CA}}
	\\
	\headerrow
		{\emph{Software Engineering Intern}}
		{\emph{May -- August 2013}}
	\begin{itemize*}
		\item Python/Django API endpoints, made tests 15\% faster, moved search backend from Solr to ElasticSearch.
	\end{itemize*}

	\item
	\headerrow
		{\textbf{University of Pennsylvania}}
		{\textbf{Philadelphia, PA}}
	\\
	\headerrow
		{\emph{Teaching Assistant \& Head Teaching Assistant}}
		{\emph{Spring 2012 -- Spring 2015}}
	\begin{itemize*}
		\item \href{https://www.seas.upenn.edu/~cis120/current/}{CIS 120} (OCaml, Java): 4 semesters. \href{http://www.seas.upenn.edu/~cis160/current/}{CIS 160} (Math Fdns of CS): Fall 2014 head TA, leading a staff of 18.
		\item \href{http://www.seas.upenn.edu/~cis121/current/}{CIS 121} (Data Structures \& Algorithms): Spring 2015 head TA, leading a staff of 20 \& designing new programming assignments. Received student feedback ratings of 3.6 (out of 4), vs. average of 2.7.
	\end{itemize*}

\end{itemize}

\hrule
\vspace{-0.7em}
\subsection*{Education}

\begin{itemize}
	\parskip=0.0em

	\item
	\headerrow
		{\textbf{University of Pennsylvania}}
		{\textbf{Philadelphia, PA}}
	\\
	\headerrow
		{\emph{School of Engineering \& Applied Science, BSE, Networked and Social Systems}}
		{\emph{Fall 2011 -- Spring 2015}}
	\begin{itemize*}
		\item First graduating class of new CS variant focused toward math, the internet, distributed systems, network theory; coursework included AI, Algorithms, Algo. Game Theory, Databases, Functional Programming, Compilers, Crowdsourcing, Cloud Computing, Networked Systems, Theory of Networks, Network Security, Cryptography, Formal Linguistics, Stochastic Systems, Probability, Linear Algebra, Discrete Math
		\item Lead organizer of \href{http://pclassic.org}{PClassic}, a high school programming contest with 300+ participants per semester.
		\item Core organizer \& MC of \href{http://pennapps.com}{PennApps} hackathon, \href{http://cis.upenn.edu/about-people/ta-hall-of-fame.php}{Penn CIS TA hall of famer}, developer with \href{http://pennlabs.org/}{PennLabs}.
		\item \href{http://voidultimate.com/}{Penn Ultimate} player, \href{http://www.pennathletics.com/sport/c-track}{Penn Track \& Field} long jumper. GPA: 3.7 in-major, 3.6 overall.
	\end{itemize*}

\end{itemize}

\hrule
\vspace{-0.7em}
\subsection*{Projects}

\begin{itemize}
	\parskip=-0.1em

	\item
	\href{https://github.com/bee-queue/bee-queue}{Bee-Queue}: Simple, fast, robust Redis-backed task queue for Node.js, built to beat perf. of existing alternatives. 

	\item
	\href{https://github.com/LewisJEllis/awesome-lua}{awesome-lua}: High-quality compilation of the modern Lua ecosystem; 2000+ stars on GitHub, listed on \href{https://www.lua.org/start.html}{Lua.org}.

	\item
	Bruce API: Backend enabling secure and scalable execution of untrusted code submissions, with frontends for programming contests and homework grading. Placed 4th out of ~30 teams in Penn CIS Senior Design.

	\item
	PSPNet, PSPTD (2008): Text-based web browser, tower defense game, written in Lua for the PlayStation Portable.
\end{itemize}


\hrule
\vspace{-0.7em}
\subsection*{Skills, Technologies, \& Interests}

\begin{indentsection}{\parindent}
\hyphenpenalty=1000
\begin{description*}
	\item[Strong with:] JavaScript, Node.js, GraphQL, Lua, Python, Redis, SQL, VSCode, Git, Docker, AWS, WebAppSec
	\item[Familiar with:] TypeScript, OCaml, Haskell, Common LISP, OpenResty, ElasticSearch, C, HTML/CSS, \LaTeX
	\item[Presentations:] \href{https://speakerdeck.com/lewisjellis/robust-error-handling-in-node-dot-js}{Robust Error Handling in Node.js}, \href{https://speakerdeck.com/lewisjellis/idiosyncrasies-of-nan-v2}{Idiosyncrasies of NaN}, \href{https://github.com/LewisJEllis/eslint101}{ESLint talk}, \href{https://github.com/LewisJEllis/nodejs-workshops}{Node.js intro workshops}
	\item[Hobbies/Interests:] Math \& trivia nerd: \href{http://mathcamp.org/}{Canada/USA Mathcamp}er, \href{https://science.energy.gov/wdts/nsb/about/}{National Science Bowl}er. Card games, board games. Club-level Ultimate player. Cars, basketball, cubing, climbing, teaching, biking, skiing, penguins.
\end{description*}
\end{indentsection}

\end{document}
